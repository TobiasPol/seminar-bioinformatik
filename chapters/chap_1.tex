\chapter{Einleitung} \label{chap:einleitung}

\section{Vorstellung der Flaviviridae-Virusfamilie}
\label{sec:vorstellung-der-flaviviridae-virusfamilie}

Die Familie der Flaviviridae umfasst eine große und vielfältige Gruppe von einzelsträngigen RNA-Viren positiver Polarität, die sowohl Menschen als auch eine Vielzahl von Tieren infizieren können \autocite{Simmonds2017}. Zu den klinisch und ökologisch bedeutsamsten Vertretern gehören Erreger schwerwiegender Krankheiten wie Dengue-Fieber, Gelbfieber, Zika-Virus-Erkrankung und Hepatitis C \autocite{Mackenzie2004}. Die Klassifikation der Flaviviridae unterteilt sich in vier Hauptgattungen: Flavivirus, Pestivirus, Pegivirus und Hepacivirus \autocite{Simmonds2017}. Neuere Entdeckungen, wie die Jingmenviren und großen Genom-Flaviviren (Large Genomes Flaviviruses, LGFs), haben die evolutionäre und ökologische Diversität dieser Familie erheblich erweitert \autocite{shiDivergentVirusesDiscovered2015}.

Eine der zentralen Eigenschaften der Flaviviridae ist die Nutzung von Glycoproteinen zur Vermittlung des Viruseintritts in Zielzellen \autocite{Mukhopadhyay2005}. Diese Proteine sind essenziell für die Bestimmung der Wirtsspezifität, des Gewebetropismus und der Pathogenese \autocite{Rey1995}. Trotz ihrer Bedeutung bestehen noch erhebliche Lücken im Verständnis ihrer Struktur und Funktion, insbesondere bei nicht-klassifizierten Mitgliedern der Flaviviridae \autocite{mifsudMappingGlycoproteinStructure2024}.

\section{Bedeutung der Glycoprotein-Struktur für Evolution und Pathogenese}
\label{sec:bedeutung-der-glycoprotein-struktur-fuer-evolution-und-pathogenese}

Die Glycoproteine der Flaviviridae sind hochinteressant in der Virusbiologie, z.B. sind sie Schlüsselstrukturen für die Interaktion mit dem Immunsystem des Wirts \autocite{Heinz2012}. Sie umfassen hauptsächlich Klasse-II-Fusionsproteine, die für die Membranfusion verantwortlich sind, wie das E-Glycoprotein der Flaviviren \autocite{kuhnStructureDengueVirus2002}. Darüber hinaus existieren potenziell neue und unbekannte Mechanismen, wie sie bei Hepaciviren und Pegiviren im E1/E2-Komplex gefunden wurden \autocite{Lavie2017}. Unterschiede in diesen Glycoproteinen spiegeln evolutionäre Anpassungen wider, die durch Mutationen, Rekombinationen und horizontalen Gentransfer vorangetrieben wurden \autocite{Weaver2009}.

Durch die Untersuchung der Glycoprotein-Divergenz können wichtige Einblicke in die Mechanismen gewonnen werden, die die ökologische Nischenanpassung und das Pathogenitätsprofil der Flaviviridae formen \autocite{mifsudMappingGlycoproteinStructure2024}. Insbesondere die Identifikation struktureller Merkmale, wie der hydrophoben Fusion-Loop-Region oder der transmembraner Domänen, liefert Hinweise auf die evolutionären und funktionellen Treiber dieser Proteine \autocite{Modis2004}.

\section{Zielsetzung und Vorgehensweise}
\label{sec:zielsetzung-und-vorgehensweise}

Diese Arbeit hat zum Ziel, durch eine umfassende Analyse der Arbeit von \textit{Mifsud, J.C.O., Lytras, S., Oliver, M.R. et al.} und dessen Methoden die Erkenntnisse über die evolutionäre Geschichte und Pathogenese der Flaviviridae gewinnen. Basierend auf dem Ansatz der referenzierter Arbeit werden phylogenetische Analysen, Proteinstrukturvorhersagen mittels ColabFold und ESMFold sowie Homologiesuchen mit Foldseek analysiert.

Die Arbeit ist wie folgt gegliedert: Zunächst werden in Kapitel 2 die eingesetzten bioinformatischen Verfahren und Werkzeuge beschrieben. In Kapitel 3 werden die wichtigsten Erkenntnisse zu phylogenetischen Beziehungen und Proteinstrukturen dargestellt. Kapitel 4 interpretiert die Ergebnisse hinsichtlich ihrer Implikationen für die Evolution und Pathogenese der Flaviviridae und diskutiert methodische Limitierungen sowie Perspektiven für zukünftige Forschung. Abschließend fasst Kapitel 5 die zentralen Ergebnisse zusammen und gibt einen Ausblick auf mögliche Anwendungen und zukünftige Fragestellungen.