\chapter{Mathematische Formeln} \label{chap:mathematische-formeln}

\section{Multiple Sequence Alignment mit MAFFT}

MAFFT ist ein Tool zur Erstellung von Multiplen Sequenzalignments (\gls{msa}). Die Qualität eines Alignments wird durch eine Punktzahl $\mathcal{S}$ bewertet, die die Übereinstimmung zwischen den Sequenzen quantifiziert:

\begin{align}
    \mathcal{S} = \sum_{i=1}^n \sum_{j=i+1}^n w_{ij} \cdot \text{Score}(i, j)
\end{align}

Hierbei ist $w_{ij}$ ein Gewichtungsfaktor, der die Relevanz des Vergleichs zwischen Sequenz $i$ und Sequenz $j$ angibt, und $\text{Score}(i, j)$ ist die Ähnlichkeit zwischen den beiden Sequenzen basierend auf Substitutionsmatrizen wie BLOSUM oder PAM.

Um Ähnlichkeiten zwischen Sequenzen effizient zu berechnen, transformiert MAFFT die Sequenzen mittels Fourier-Transformation in das Frequenzspektrum:

\begin{align}
    \mathcal{F}(k) = \sum_{n=0}^{N-1} x(n) \cdot e^{-i \frac{2\pi k n}{N}}
\end{align}

Dabei ist $\mathcal{F}(k)$ das Frequenzspektrum der Sequenz, $x(n)$ die diskrete Funktion der Sequenz an Position $n$, $N$ die Länge der Sequenz und $i$ die imaginäre Einheit.

Die Distanz zwischen Sequenzen wird durch eine Distanzmatrix $\mathcal{D}(i, j)$ berechnet:

\begin{align}
    \mathcal{D}(i, j) = 1 - \frac{\sum_{k=1}^L \delta(x_k^i, x_k^j)}{L}
\end{align}

Hier ist $\delta(x_k^i, x_k^j)$ eine Indikatorfunktion, die 1 ist, wenn die Aminosäuren an Position $k$ in Sequenz $i$ und $j$ identisch sind, sonst 0. $L$ ist die Länge der Ausrichtung.

\section{Maximum-Likelihood-Methode und Substitutionsmodell}

Die Maximum-Likelihood-Methode zielt darauf ab, die Baumtopologie $T$ zu finden, die die Wahrscheinlichkeit der beobachteten Daten $D$ maximiert, gegeben ein Modell $M$:

\begin{align}
    \mathcal{L}(M) = P(D \mid M) = \prod_{i=1}^{n} P(d_i \mid M)
\end{align}

Dabei ist $\mathcal{L}(M)$ die Likelihood des Modells, und $P(d_i \mid M)$ die Wahrscheinlichkeit der Daten an Position $i$, gegeben das Modell $M$.

Das verwendete Substitutionsmodell \gls{gtrig} berücksichtigt unterschiedliche Substitutionsraten zwischen Nukleotiden, einen Anteil invarianter Positionen und gamma-verteilte Rate-Heterogenität. Die genaue Formulierung des Modells beinhaltet komplexe mathematische Gleichungen zur Beschreibung der Übergangswahrscheinlichkeiten zwischen Nukleotiden.

\section{Berechnung der Root-Mean-Square Deviation (RMSD)}

Die Root-Mean-Square Deviation (\gls{rmsd}) ist ein Maß für die durchschnittliche Distanz zwischen entsprechenden Atomen zweier Proteinstrukturen nach optimaler Überlagerung:

\begin{align}
    \text{RMSD} = \sqrt{\frac{1}{N} \sum_{i=1}^N \| \mathbf{p}_i - \mathbf{q}_i \|^2}
\end{align}

Hierbei sind $\mathbf{p}_i$ und $\mathbf{q}_i$ die Ortsvektoren des $i$-ten Atoms in den beiden zu vergleichenden Strukturen, und $N$ ist die Gesamtzahl der Atome (oder C$_\alpha$-Atome) im Vergleich. $\| \cdot \|$ bezeichnet die euklidische Norm.

Die RMSD-Berechnung ermöglicht die Quantifizierung der strukturellen Ähnlichkeit zwischen Proteinmodellen und dient als Kriterium für die Qualität der Strukturüberlagerung.

