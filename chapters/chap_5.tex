\chapter{Schlussfolgerung} \label{chap:schlussfolgerung}

Die vorliegende Arbeit liefert neue Erkenntnisse zur evolutionären Geschichte und strukturellen Diversität der Glycoproteine innerhalb der Flaviviridae-Familie. Durch die Kombination von phylogenetischen Analysen, Proteinstrukturvorhersagen mittels \gls{colabfold} und \gls{esmfold} sowie strukturellen Homologiesuchen mit \gls{foldseek} konnten sowohl konservierte als auch variable Merkmale der Glycoproteine identifiziert und charakterisiert werden \autocite{mifsudMappingGlycoproteinStructure2024}.

\section{Wichtige Ergebnisse und Implikationen} \label{sec:zusammenfassung-der-ergebnisse}

Die phylogenetische Analyse unterteilte die Flaviviridae in drei Hauptkladen: \textit{Flavivirus}, \textit{Pegivirus/Hepacivirus} und \textit{Pestivirus/Jingmenvirus/LGFs}. Diese Aufteilung korreliert eng mit den strukturellen Eigenschaften der Glycoproteine und verdeutlicht die evolutionäre Verknüpfung zwischen genetischer Sequenzkonservierung und funktioneller Diversifikation. Konservierte Regionen wie die hydrophoben Fusion-Loops und transmembranen Domänen wurden über alle Hauptkladen hinweg nachgewiesen und unterstreichen ihre essentielle Rolle bei der Membranfusion und dem Viruseintritt \autocite{Modis2004}. Aufgrund dieser funktionellen Konservierung bieten sie vielversprechende Zielstrukturen für breit wirksame antivirale Therapien und Impfstoffentwicklungen \autocite{Fernandez2018}.

Gleichzeitig wurden in den Glycoproteinen signifikante strukturelle Variationen festgestellt, die adaptive Mechanismen widerspiegeln. Insbesondere die E1/E2-Komplexe der Hepaciviren und Pegiviren zeigten eine erhebliche Diversität in der räumlichen Anordnung der Oberflächenregionen, was auf eine mögliche Anpassung an unterschiedliche Wirte und Immunabwehrmechanismen hinweist \autocite{Vieyres2013, Lavie2017}. Diese Variationen eröffnen neue Ansätze für gattungsspezifische antivirale Strategien, die auf variablen Regionen wie den Glycosylierungsstellen basieren.

Methodisch demonstrierte die Arbeit die Leistungsfähigkeit moderner bioinformatischer Werkzeuge zur Strukturvorhersage und Homologieerkennung. Die Kombination von ColabFold und ESMFold lieferte detaillierte Einblicke in die dreidimensionalen Strukturen der Glycoproteine, offenbarte jedoch auch die Notwendigkeit experimenteller Validierung. Dies gilt insbesondere für hochdivergente Proteine, bei denen die Vorhersagegenauigkeit durch die begrenzte Verfügbarkeit homologer Referenzdaten eingeschränkt sein könnte.

\section{Persönliche Kritik und Reflexion} \label{sec:persoenliche-kritik}

Die zugrundeliegende Arbeit hebt hervor, wie interdisziplinäre Ansätze, insbesondere die Anwendung bioinformatischer Werkzeuge, einen entscheidenden Beitrag zum Verständnis der Flaviviridae-Familie leisten können. Dennoch gibt es wesentliche Aspekte, die einer kritischen Reflexion bedürfen.

Die Nutzung moderner Werkzeuge wie ColabFold, ESMFold und Foldseek zeigt, wie maschinelles Lernen die Proteinstrukturvorhersage und phylogenetische Analyse vorantreiben. Diese Technologien demonstrieren nicht nur eine hohe Effizienz und Präzision, sondern auch die Fähigkeit, komplexe biologische Fragestellungen in einem breiten Maßstab zu adressieren. Besonders positiv hervorzuheben ist die konsistente Nutzung standardisierter bioinformatischer Pipelines, die sowohl Reproduzierbarkeit als auch Skalierbarkeit gewährleisten. Durch die stetige Weiterentwicklung dieser Methoden könnten zukünftige Studien noch tiefere Einblicke in die Struktur-Funktions-Beziehungen der Glycoproteine ermöglichen.

Trotz der Fortschritte in der Proteinstrukturvorhersage bleibt die Abhängigkeit von in silico Methoden eine zentrale Limitation. Die Vorhersagegenauigkeit ist bei hochdivergenten Glycoproteinen, wie sie in dieser Arbeit untersucht wurden, eingeschränkt, insbesondere wenn homologe Referenzstrukturen fehlen. Dies unterstreicht die Notwendigkeit, maschinelle Lernmodelle kontinuierlich durch experimentelle Daten zu validieren und zu verbessern. Dise ist eine Herausforderung und zugleich eine Chance, durch erweiterte Trainingsdatensätze die Vorhersagekraft zu steigern.

Diese Arbeit verdeutlicht eindrucksvoll, wie stark Methoden des maschinellen Lernens das Verständnis komplexer biologischer Phänomene erweitern können. Die Arbeit hat mir gezeigt, wie wichtig die Verbindung von technischen Kompetenzen mit domänenspezifischem Wissen ist, um interdisziplinäre Herausforderungen zu bewältigen. In Zukunft wird eine noch engere Zusammenarbeit zwischen Bioinformatikern, Virologen und Strukturbiologen entscheidend sein, um die vielfältigen Facetten der Virus-Familien zu entschlüsseln und neue Therapieansätze zu entwickeln.

\section{Zukünftige Forschungsrichtungen} \label{sec:zukuenftige-forschung}

Basierend auf den Ergebnissen dieser Arbeit ergeben sich mehrere zentrale Forschungsrichtungen für die Zukunft. Ein vorrangiges Ziel sollte die experimentelle Validierung der vorhergesagten Glycoprotein-Strukturen sein. Methoden wie Kryo-Elektronenmikroskopie oder Röntgenkristallographie könnten die Präzision der in silico-Modelle bestätigen und zusätzliche Einblicke in die funktionelle Bedeutung der konservierten und variablen Regionen liefern \autocite{Callaway2020}.

Erweiterte phylogenetische Analysen, die zusätzliche genetische Marker einbeziehen, könnten die Auflösung der evolutionären Beziehungen weiter verbessern. Die Integration von Metagenomik-Daten bietet zudem die Möglichkeit, bisher unbekannte Viruslinien zu identifizieren und ein umfassenderes Verständnis der Diversität und Adaptionsmechanismen der Flaviviridae zu gewinnen. Dies wäre besonders wertvoll, um zoonotische Risiken und die Anpassung von Viren an neue Wirtsorganismen besser abschätzen zu können.

Ein weiterer Schwerpunkt zukünftiger Forschung sollte auf die therapeutische Anwendung der Erkenntnisse gelegt werden. Die konservierten Regionen, wie die Fusion-Loops und transmembranen Domänen, stellen ideale Angriffspunkte für breit wirksame antivirale Wirkstoffe und Impfstoffe dar. Gleichzeitig könnten gattungsspezifische strukturelle Unterschiede zur Entwicklung gezielter antiviraler Strategien genutzt werden, die spezifisch auf variablen Regionen der Glycoproteine basieren.

Schließlich eröffnet die Weiterentwicklung bioinformatischer Methoden vielversprechende Perspektiven. Die Kombination von maschinellem Lernen mit experimentellen Daten könnte die Vorhersagegenauigkeit für schlecht charakterisierte Proteine erheblich verbessern und neue Möglichkeiten zur Analyse funktioneller Proteindomänen bieten \autocite{Senior2020}. Verbesserungen in der Effizienz und Präzision dieser Modelle würden insbesondere bei hochdivergenten Glycoproteinen von großer Bedeutung sein.